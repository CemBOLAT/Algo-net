\documentclass[12pt,a4paper]{article}
\usepackage{amsmath, amssymb, amsthm}
\usepackage{geometry}
\geometry{margin=1in}
\usepackage{setspace}
\setstretch{1.2}

\title{Cover Shelter Design}
\author{Mehmet Melikşah Çalışkan \and Cemal Bolat}
\date{October 2025}

\begin{document}

\maketitle

\section{Input Variables}

The following sets and parameters are defined as input data to the model:

\begin{itemize}
    \item $V$: set of vertices (discrete container locations), $|V|=n$.
    \item $T$: set of container types.
    \item $r \in T$: residential container type.
    \item $D_t$: Distance limit (Manhattan-Euclidean) between residential $r$ and non-residential type $t \in T \setminus\{r\}$.
    \item $A_t$: Required size of a group (number of containers) for type $t \in T \setminus \{r\}$.
    \item $\Delta_t$: Maximum allowable diameter (distance threshold) for type $t \in T \setminus \{r\}$.
    \item $d(u,v)$: Manhattan-Euclidean shortest distance between vertices $u,v \in V$. (The shortest real life path between vertices) 
    \item $h(u,v)$: Euclidean distance between vertices $u,v\in V$ (used in the clustering penalty).
    \item $G_t$: Upper bound on the number of possible groups for type $t$.
    \item $\text{Allowed}_t$: Subset of vertices feasible for type $t$ (based on component pruning).
    \item $\text{TooFar}_t = \{(u,v) \in \text{Allowed}_t \times \text{Allowed}_t \mid d(u,v) > \Delta_t\}$.
\end{itemize}


\vspace{1em}
\textbf{Neighborhood sets (precomputed):}

\[
S_{v,t} = \{\, u \in V \mid u \neq v \text{ and } d(u,v) \le D_t \,\},
\quad \forall v\in V,\ \forall t\in T\setminus\{r\}.
\]

\textbf{Explanation:}
For each vertex $v \in V$ and non-residential container type $t \in T \setminus \{r\}$,
the set $S_{v,t}$ represents all neighboring vertices that lie within the
maximum allowable distance $D_t$ for that type.
These neighborhoods are used to evaluate the ``rainbow coverage'' constraint,
ensuring that each residential vertex is surrounded by nearby containers of different types.

\vspace{1em}
\textbf{Allowed vertices for type size (precomputed):}


\[
\text{Allowed}_t
= 
\Bigl\{\, 
v \in V
\ \Big|\
\exists\, C \subseteq V :
\begin{array}{l}
C \text{ is a connected component of } G_t = (V, E_t), \\[4pt]
E_t = \{\, (u,v) \in V \times V \mid d(u,v) \le \Delta_t \,\}, \\[6pt]
v \in C \text{ and } |C| \ge A_t
\end{array}
\Bigr\},
\quad 
\forall t \in T \setminus \{r\}.
\]

\textbf{Explanation:} 
For each non-residential container type $t$, the graph $G_t$ is constructed by connecting 
vertices whose pairwise distance does not exceed the type-specific diameter limit $\Delta_t$.
Each connected component $C$ of $G_t$ represents a spatially contiguous cluster of vertices.
The set $\text{Allowed}_t$ then consists of all vertices that belong to at least one such component 
containing at least $A_t$ vertices, ensuring that every feasible placement 
can support a full group of size $A_t$ for container type $t$.

\vspace{1.5em}
\textbf{Too-distant vertex pairs (precomputed):}


\[
TooFar_t = \{(u,v) \in Allowed_t \times Allowed_t \mid d(u,v) > \Delta_t\}.
\]

\textbf{Explanation:}
For each non-residential container type $t$, 
the set $\text{TooFar}_t$ contains all unordered vertex pairs $(u,v)$ 
whose mutual distance exceeds the maximum diameter $\Delta_t$ allowed for type $t$.
These precomputed pairs are used to impose compactness constraints,
preventing two vertices that are too far apart from belonging to the same container group.

\section{Decision variables (core model)}

Primary assignment variables:
\[
X_{v,t} =
\begin{cases}
1, & \text{if vertex } v \text{ is assigned type } t,\\[4pt]
0, & \text{otherwise.}
\end{cases}
\qquad \forall v\in V,\ \forall t\in T.
\]

\[
Y_{g,t,v} = 
\begin{cases}
1, & \text{if vertex $v$ belongs to group $g$ of type $t$,}\\
0, & \text{otherwise.}
\end{cases}
\quad \forall t \in T \setminus \{r\}, g = 1,\dots,G_t,v \in \text{Allowed}_t.  
\]

\[
U_{g,t} =
\begin{cases}
1, & \text{if group $g$ of type $t$ is activated,}\\[3pt]
0, & \text{otherwise;}
\end{cases}
\qquad \forall g = 1,\dots,G_t,\ \forall t \in T \setminus \{r\}.
\]

All decision variables are binary:
\[
X_{v,t},\ Y_{g,t,v},\ U_{g,t} \in \{0,1\}.
\]

\section{Objective}

Maximize the total number of residential containers:
\[
\max \sum_{v \in V} X_{v,r}.
\]

\section{Core constraints}

\begin{enumerate}
    \item \textbf{Single type per vertex:}
    \[
    \sum_{t\in T} X_{v,t} \le 1 \qquad \forall v\in V.
    \]
    Each vertex can host at most one container type.
    \item \textbf{Neighborhood Coverage (Rainbow Constraint):}
    \[
    \sum_{u\in S_{v,t}} X_{u,t} \ge X_{v,r}
    \qquad \forall v\in V,\ \forall t\in T\setminus\{r\}.
    \]
    If a vertex v is residential, it must have at least one neighboring vertex of each
non-residential type within distance Dt
.
    \item \textbf{Group Size Consistency:}
    \[
    \sum_{v \in \text{Allowed}_t} Y_{g,t,v} = A_t \, U_{g,t} 
    \qquad \forall t \in T \setminus \{r\}, \forall g = 1,\dots,G_t.
    \]

    Each activated group of type $t$ must contain exactly $A_t$ vertices:

    \item\textbf{Linking Groups and Vertex Assignment}
    \[
    X_{v,t} = \sum_{g \in \text{ValidGroups}_{v,t}} Y_{g,t,v} 
    \qquad \forall v \in V, \forall t \in T \setminus \{r\},
    \]
    where 
    \[
    \text{ValidGroups}_{v,t} =
    \begin{cases}
    \{1,\dots,G_t\}, & \text{if } v \in \text{Allowed}_t,\\[3pt]
    \varnothing, & \text{otherwise.}
    \end{cases}
    \]

    A vertex $v$ is assigned type $t$ if and only if it belongs to one active group of type $t$:

    \item\textbf{Diameter Restriction}
    \[
    Y_{g,t,u} + Y_{g,t,v} \le 1
    \qquad 
    \forall t \in T \setminus \{r\}, \forall g = 1,\dots,G_t, 
    \forall (u,v) \in \text{TooFar}_t.
    \]
    Any two vertices that are farther apart than $\Delta_t$ cannot belong to the same group:
    \item\textbf{ Sequential Symmetry Breaking}
    \[
    U_{g,t} \ge U_{g+1,t} 
    \qquad \forall t \in T \setminus \{r\}, \forall g = 1,\dots,G_t - 1.
    \]
    Groups of each type are activated in order to eliminate symmetric solutions:
\end{enumerate}

\textbf{Explanation:}  
These constraints ensure the logical and spatial feasibility of the model:
\begin{enumerate}
    \item Each vertex hosts exactly one container type, preventing overlap between different functions.
    \item Residential vertices are supported by a diverse set of neighboring container types within predefined distance limits (rainbow coverage).
    \item Each non-residential container group maintains the required number of units, ensuring consistent group sizes.
    \item Vertex-type assignments and group memberships are linked, guaranteeing that every container of a given type belongs to exactly one active group.
    \item The diameter restriction enforces spatial compactness by preventing distant vertices from being assigned to the same group.
    \item Sequential symmetry-breaking constraints eliminate redundant solutions by activating groups in order.
\end{enumerate}

\end{document}
